\subsection{Super Mario World}

We are going to focus on a single game for level generation:
\emph{Super Mario World} (SMW). Super Mario World was released in~1990
by Nintendo for the Super Nintendo Entertainment
System~\cite{SuperMarioWorld2019}. It is a 2-dimensional platforming
game with the goal of reaching each level's exit in a limited amount
of time. Games in the platforming genre confront the player with
dangers like pits and enemies which have to be avoided by carefully
jumping and maneuvering accross them in a host of levels. The original
game features % TODO
~levels. \\
The player has access to a moveset including but not limited to
running, jumping, picking up and throwing objects, and even shooting
fireballs, flying and riding a dinosaur extending the amount of
abilities even further. To access some abilities, Mario has to pick up
power-ups usually contained in ``question mark blocks'' (``? blocks'').
Blocks are commonly activated by hitting them from below. During a
level, the player may collect various items like the power-ups
mentioned above (which also let Mario take one extra hit before he is
defeated) as well as coins and one-up mushrooms which increase the
amount of lifes (tries) available to the player. \\
Usually, when Mario lands on top of an enemy, the enemy is defeated.
There are some enemies which Mario cannot safely land on or cannot
defeat by jumping on top of them. Those enemier require a different
trick like throwing something at them or hitting a block they are
standing on from below. \\
Sometimes, the player may even have to use an enemy to complete a
level. For example, the turtle-like ``Koopas'' get knocked out of
their shell before being defeated. The shell has a variety of uses
like activating an unreachable block, defeating enemies and even
giving access to mid-air jumps by throwing and subsequently landing on
the shell.
% TODO figure for koopa

The next two sections will give an in-depth look into why Super Mario
World was chosen as a machine learning problem and just how complex
the levels are. We will then give a more low-level overview into how
the levels work and what one has to look out for when creating a
database suitable for feeding into a machine learning model.

\subsubsection{Hacking Scene}

At the time of writing, Super Mario World is almost 30 years old. Due
to both its age and success, it has been able to gain a large
following of people not only interested in playing the game but also
extending it. As with many other~-- especially retro~-- games, there
is a read-only memory (ROM) hacking scene around Super Mario World. In
this thesis, ``ROM'' will refer to the digital form of the Super Mario
World ROM cartridge that would usually be inserted into the Super
Nintendo Entertainment System; like the data that is burned onto a CD.
A lot of tools have been created to manipulate Super Mario World; even
in ways that the original creators hadn't programmed. People are able
to create their own levels and share them with the masses by uploading
a patch file that has to be applied to the original game ROM\footnote{Usually
the American version; CRC32 checksum \texttt{a31bead4}.}, thereby
avoiding copyright infringement as sharing the ROM, Nintendo's
protected property, is illegal.

Due to the refined tooling available, users do not need to program in
65C816 assembly\footnote{The Super Nintendo Entertainment System's 16
  bit microprocessor is based on the 65C816 microprocessor.} to modify
the game. Instead, even non-technical users are able to use one of the
oldest tools to modify Super Mario World: \emph{Lunar
  Magic}~\cite{FuSoYaNicheLunar}. It features a GUI and has optional
features that extend Super Mario World with new functionality such as
larger levels, custom coded tiles, and more. The creator of Lunar
Magic, \emph{FuSoYa}, modified the existing software to support both
dumping and reconstructing levels, making this work possible due to
the reduced workload of parsing the data from the binary ROM. The
features are currently only available in a private build (based on
Lunar Magic version 3.04) that will be used throughout the thesis to
(1) parse level data from existing ROMs and to (2) write newly
generated levels back into the original Super Mario World ROM.

With Lunar Magic as a user friendly tool granting creators all the
freedom they could ever desire to design their own Mario levels,
communities around ROM hacking emerged. One of those communities is
\emph{SMW Central}~\cite{SMWCentralYour}, hosting most of the
available Super Mario World ROM hacks. Due to both SMW Central's
filtering capabilities and its quality of content, it was the source
for all levels used in this thesis (excluding the ones in the original
ROM). All the hacks used were obtained from SMW Central where their
respective authors (who are not necessarily related to SMW Central)
uploaded them.

Being able to filter hacks was important for two reasons. One being
that programmed custom behaviour was out of the scope of this work,
requiring both parsing the binary 65C816 assembly code and then
understanding its every behaviour. So we used the ``vanilla'' tag to
exclude hacks using behaviour not in the original (``vanilla'') game.
Also, to assert some form of quality in the dataset, only hacks with a
rating greater than or equal to 3.0 were downloaded (our rating is the
mean of all individual ratings, ranging from 1.0 to 5.0 and including
\emph{None} for no individual ratings). All in all, over 17000 unique
levels in over 300 hacks were obtained thanks to the community keeping
a masterpiece alive and fresh for not only all the years since the
game's release but also for many years to come. While this amount of
data is not a lot relative to what deep learning models usually train
on in 2019 (with data points in the millions, even approaching
billions), it may be enough for our models to get a grasp on what
makes a level enjoyable.

\subsubsection{Levels}

We already assessed that Super Mario World is a complex game with many
possibilities both in play and creative freedom in level design due to
the vast number of interactions in the virtual world. We will now fill
in some missing gaps of what makes up a level.

%TODO figures for goal and others
To give the player an easier or less frustrating time, most levels
contain a \emph{midway point}. When activated (achieved by touching
the midway point), Mario will spawn at that point instead of the usual
level entry point in case he is defeated. Most levels end when Mario
reaches the goal post. Some levels end when Mario touches an orb while
others end when a key is transported to and inserted into a keyhole.
Finally, there are boss levels that are completed by defeating a boss.
We will, however, not consider these.

Each level may be connected to a number of other levels. The game
achieves this through two systems which will be explained in the
paragraph on layers on page~\pageref{par:layers}. The high-level
connection between two levels is given by tiles with special behaviour
that can be entered by Mario. These tiles are called ``exit-enabled''
tiles. The most common are doors and pipes (only some pipes are
exit-enabled; those cannot be visually distinguished from those which
aren't).

Levels can have underwater regions through which Mario will have to
swim, removing his ability to jump on enemies to defeat them. Some
levels are vertical, growing in height rather than width. Other levels
are covered in darkness, where light is only emitted in a small radius
around Mario while others automatically scroll the level forward,
forcing the player to react more quickly and not get stuck on a wall
(when the level scrolls beyond Mario, he is defeated). Finally, a
level may contain ``blue P switches'' that temporarily turn coins into
blocks and certain blocks into coins, granting new paths to the player
that are possibly required to reach the goal. A very similar tile is
the ``grey P switch'' which spawns coins and special doors.

All these different tiles can be combined in any way imaginable. Most
of those combinations in the high-dimensional level space are
non-playable nonsense. However, as described in
section~\ref{sec:hacks}, due to the large amount of possible~-- and,
more importantly, enjoyable~-- interactions between all the different
pieces contained in the original game, a seemingly infinite number of
unique, fun levels can be created.
\medskip

We are now going to leave the high-level view of levels and take a
look at how everything works together on a lower level: \\
Levels are made up of \emph{screens}; matrices of the same size. Each
screen is made up of two subscreens; in horizontal levels, subscreens
are stacked vertically and the whole, combined screens are
concatenated horizontally while in vertical levels, subscreen are
stacked horizontally and screens concatenated vertically. While
subscreens are important in data preparation, we will not focus on
them in this work and instead always refer only to whole screens. A
screen is a 2-dimensional slice of the level consisting of a number of
columns and rows. Screens in horizontal and vertical levels have
different sizes (even if horizontal screens are transposed), but
otherwise, screen sizes are constant over different levels. While
Lunar Magic supports non-vanilla behaviour in that screens can have
other sizes (although still a constant size per level). Like with all
non-vanilla modifications that have a direct impact on data, we are
going to ignore that feature.

Calling a screen a matrix is not the whole truth. While screens itself
are basically only areas in a level, those areas can contain many
different layers of other types of objects.

\paragraph{Layers}
\phantomsection
\label{par:layers}

While we will explain ``layers'' in this section, in the hacking
community, the term ``layer'' refers to both interactive and graphical
tilemaps (of which there are three of). When we refer to the second
meaning, we will explicitly list a number behind the layer (for
example ``layer 2''). \\
Disregarding the previously explained screens, levels consist mainly
of five different types of data which each correspond to a file dumped
by Lunar Magic. The types and a short description are listed here:
\begin{enumerate}
\item \emph{Tiles} are most of the static blocks in a level. They
  usually make up the ground, platforms, walls and ceiling of a level
  (if present). Also, most collectibles that are immediately visible
  in the level are a tile; examples are coins or moons\footnote{Moons
    give the player three lifes upon collection.}. Some special tiles
  like doors and some pipe tiles are exit-enabled, meaning they lead
  out of a level (exits are described below). Even non-interactive
  background objects like arrow signs or bushes that only exist for
  aesthetical reasons are tiles. In the hacking community, this layer
  of tiles is also called ``layer 1''. \\
  As mentioned at the beginnig of this paragraph, tiles may exist on
  two layers in the level. Layer 2 may contain interactive objects or
  just non-interactive graphical background data. A special property
  of layer 2 tiles is that they can move if desired (how that is
  achieved will be described in a bit). When layer 2 is used for
  interactive tiles, it cannot act as a background tilemap. Therefore,
  a third (strictly non-interactive) layer exists for the sole purpose
  of a background tilemap which is usually in layer 2. This layer 3 is
  not related to tiles in any way. For simplicity, we ignore all data
  in layers 2 and 3.
\item \emph{Metadata} is any data that changes the behaviour of a
  level without belonging to a certain position in the level. For
  example, which tileset (that is, the graphics) or which background
  music is used in a level is decided by the metadata. A level's
  metadata also contains its entry points. We will describe metadata
  in detail in its paragraph on page~\pageref{par:metadata}.
\item \emph{Sprites} are usually moving or movable objects that
  interact with Mario and each other in a certain way. For example,
  enemies and even goal objects like the orb, key and keyhole
  (described in section~\ref{sec:levels}) are sprites (while the goal
  post is a tile, the moving tape Mario can cross through is also a
  sprite). \\
  Sprites may also have other effects on the level; for example,
  sprites control auto-scroll behaviour in a level or the movement of
  layer 2 tiles. Sprites may also affect a level's lighting or
  background objects. When Mario reaches a screen with one of those
  special sprites, the behaviour will activate.
\item \emph{Exits} connect one level to another. An exit is
  screen-based, meaning it will be used by all exit-enabled tiles in
  the screen the exit is contained in. There are two types of exits:
  \emph{Screen exits} send Mario to the entry point of another level
  while \emph{secondary exits} can send Mario to a designated point in
  the exit-level, described below.
\item \emph{Secondary entrances} are the destinations of secondary
  exits. These destinations cannot lie anywhere; they have certain
  pre-determined positions based on a table of positional values and
  based on whether they are in the first or second subscreen of the
  screen the secondary entrance lies in. Any number of secondary exits
  may point to the same secondary entrance.
\end{enumerate}

The data dumped by Lunar Magic is split into five different files with
the following extensions and representation:
% TODO this is crap
\begin{description}
\item[map] Tiles; matrix-like.
\item[ent] Metadata; struct-like.
\item[sp] Sprites; list-like format.
\item[sp] Sprite data
\end{description}
Super Mario World levels 

\paragraph{Metadata}
\phantomsection
\label{par:metadata}

\paragraph{Levels in Other Games}


%%% Local Variables:
%%% mode: latex
%%% TeX-master: "../../SMWLevelGenerator"
%%% End:

