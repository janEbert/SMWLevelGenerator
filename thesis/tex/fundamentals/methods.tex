\subsection{Methods}

In this section, we will introduce the primary ideas we apply to
obtain a level generation pipeline. Before we list the methods, we
take a look at how to actually achieve our task. By analyzing inputs
of large dimensionality (the level's layers and metadata), we want to
generate similar, new levels of large dimensionality (also level
layers and metadata). This is a very complex problem, requiring
careful planning and a lot of computational power to train a model
large or refined enough to capture all the essences we need for a new,
enjoyable level. A generative adversarial network~(GAN)~\cite{} would~-- with
the present knowledge in machine learning~-- be exceptionally hard to
train on a task this large. A lack of proper computational resources
makes this even harder as training a GAN requires training two full
models in parallel. While a GAN would be perfect in theory, allowing
us to learn the dataset without a modified loss function, due to the
issues with training, we try another approach: Generating only the
first screen by the GAN and generating the rest of the level using
time series prediction methods makes training much easier, allowing us
to get a simpler solution to our problem. Although the results will
not be as good as with a GAN due to a non-adversarial loss function,
with some tuning, good results should be achievable. Also, training of
both, the GAN and the time series prediction model, should be much
easier than a single, large GAN. To generate the metadata, we use a
simple image processing model that predicts a level's metadata based
on its first screen.

The methods will not be listed in pipeline-sequential ordering (the
order in which they will be applied to obtain a generated level) but
instead we will focus on the amount of data the different methods
generate. The sequence prediction task is the most important part of
the pipeline, generating a complete level from a minimal amount of
input. To obtain the first input of the to be predicted sequence, we
use a generative model. Finally, to generate the metadata described on
page~\pageref{par:metadata}, we use an image processor that predicts
the metadata of a level by analysing its first screen~-- the output of
the generative model and input to the sequence predictor.
% TODO picture of pipeline here or maybe at end of section

\subsubsection{Sequence Prediction}

Tasked with completing an unfinished sequence, we turn our head to
time series prediction models. The models we are using are also called
sequence-to-sequence models. We will closely orient ourselves on
natural language processing techniques, specifically natural language
generation. The models we evaluate are long short-term memory
(LSTM)~\cite{hochreiterLongShorttermMemory1997} stacks (with a
fully-connected layer at the end) and transformer-based models. More
specifically, the LSTM stacks are based on
\emph{char-rnns}~\cite{andrejKarpathyCharrnn2019} and the
transformer-based models are
\emph{GPT-2}~\cite{radfordLanguageModelsAre,OpenaiGpt22019} models. As
a baseline, we implement a non-learning random model that simply
outputs sequences of either 1 or 0 based on a user-supplied chance
$p \in [0, 1]$.

A long short-term memory cell's specialty is a recurrent memory that
``remembers'' values of a sequence, depending on connection weights.
Due to this behaviour, it is well suited for both non-linear and
long-term dependencies in Mario levels as both can be captured from
any prior, relevant sequence element (for example, a keyhole at the
beginning of the level implies a key later and the other way around).

An LSTM cell is defined by the following formulae (reproduced from~\cite{hochreiterLongShorttermMemory1997}):
\begin{align*}
  y
\end{align*}

Transformer-based architectures rely on a combination of attention
functions with multiple ``heads'' with dense networks. These networks
improve upon the LSTM with parallelizability which~-- with the
required computational power for increasingly large neural networks~--
is a very important factor. Unlike LSTMs, transformers do not keep a
hidden state; they analyze the whole input at once. Due to not having
to keep a state, transformers can analyze multiple different inputs at
once, non-sequentially. This enables the parallelizability which in
turn enables faster training enabling larger models. With these
advantages, transformers should be better suited to the task. With the
GPT-2 architecture~\cite{radfordLanguageModelsAre}, transformers'
capabilities in natural language generation tasks have been shown.
While those models used an amount of parameters that could not
possibly be trained with our available resources, we hope that a
smaller, trainable model may still be marginally larger than an LSTM
with the same training time.

% TODO viel zu erklären hier

\subsubsection{Generative Methods}

For our only actual generator, we supply two models: a standard deep
convolutional generative adversarial network
(DCGAN)~\cite{radfordUnsupervisedRepresentationLearning2016} and a
Wasserstein GAN (also based on the DCGAN). Both of these models are
specially fitted with stride, padding and dilation values to obtain
the correct screen size ($27 \times 16 \times ch$, where $ch$ is the
number of channels or layers).
% TODO

\subsubsection{Image Processing}

Similar to the generative models, we also use a deep convolutional
network with stride, padding and dilation values fitted towards the
correct screen size.
% TODO


%%% Local Variables:
%%% mode: latex
%%% TeX-master: "../../SMWLevelGenerator"
%%% End:

