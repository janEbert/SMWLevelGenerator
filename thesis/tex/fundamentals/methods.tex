\subsection{Methods}

In this section, we will introduce the primary ideas we apply to
obtain a level generation pipeline. Before we list the methods, we
take a look at how to actually achieve our task. By analyzing inputs
of large dimensionality (the level's layers and metadata), we want to
generate similar, new levels of large dimensionality (also level
layers and metadata). This is a very complex problem, requiring
careful planning and a lot of computational power to train a model
large or refined enough to capture all the essences we need for a new,
enjoyable level. A generative adversarial network~(GAN) would~-- with
the present knowledge in machine learning~-- be exceptionally hard to
train on a task this large. A lack of proper computational resources
makes this even harder as training a GAN requires training two full
models in parallel. While a GAN would be perfect in theory, allowing
us to learn the dataset without a modified loss function, due to the
issues with training, we try another approach: Generating only the
first screen by the GAN and generating the rest of the level using
time series prediction methods makes training much easier, allowing us
to get a simpler solution to our problem. Although the results will
not be as good as with a GAN due to a non-adversarial loss function,
with some tuning, good results should be achievable. Also, training of
both, the GAN and the time series prediction model, should be much
easier than a single, large GAN. To generate the metadata, we use a
simple image processing model that predicts a level's metadata based
on its first screen.

The methods will not be listed in pipeline-sequential ordering (the
order in which they will be applied to obtain a generated level) but
instead we will focus on the amount of data the different methods
generate. The sequence prediction task is the most important part of
the pipeline, generating a complete level from a minimal amount of
input. To obtain the first input of the to be predicted sequence, we
use a generative model. Finally, to generate the metadata described on
page~\pageref{par:metadata}, we use an image processor that predicts
the metadata of a level by analysing its first screen~-- the output of
the generative model and input to the sequence predictor.
% TODO picture of pipeline here or maybe at end of section

\subsubsection{Sequence Prediction}

Tasked with completing an unfinished sequence, we turn our head to
time series prediction models. The models we are using are also called
sequence-to-sequence models. We will closely orient ourselves on
natural language processing techniques, specifically natural language
generation. The models we evaluate are long short term memory (LSTM)
stacks and transformer-based models. More specifically, the LSTM
stacks are based on \emph{char-rnns}~\cite{andrejKarpathyCharrnn2019}
and the transformer-based models are
\emph{GPT-2}~\cite{radfordLanguageModelsAre,OpenaiGpt22019} models. As
a baseline, we implement a non-learning random model that simply
outputs sequences of either 1 or 0 based on a user-supplied chance
$p \in [0, 1]$.



\subsubsection{Generative Methods}

% TODO


\subsubsection{Image Processing}

% TODO


%%% Local Variables:
%%% mode: latex
%%% TeX-master: "../../SMWLevelGenerator"
%%% End:

