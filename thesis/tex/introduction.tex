\section{Introduction}

Currently, AI is being used to reshape our lives. However, a common
discussion in recent times is malevolent usage of AI technology.
Generative methods especially have evolved rapidly, becoming much more
sophisticated than some years ago. With the advent of GANs being able
to be used for movie production~\cite{Face2FaceRealtimeFace} or
quality enhancements~\cite{wangESRGANEnhancedSuperResolution2018} and
more refined language models that may produce texts passing the Turing
test~\cite{TuringTest2019,radfordLanguageModelsAre}, generative
methods seem to be in a golden age.

This thesis provides a fun, artistic view into generative
possibilities. Ultimately, this project provides research perspectives
into combining different fields of machine learning, namely natural
language processing, generative methods and image processing. With a
combination of techniques used in these three fields, we generate most
of our level by prediction, avoiding the problems usually faced during
training of generative adversarial networks. \\
We present \emph{SMWLevelGenerator}, a pipeline to train models
capable of generating Super Mario World~\cite{SuperMarioWorld2019}
levels from scratch. We encourage readers to become users and try out
the provided functionality, maybe even extending the pipeline with new
models or ideas.

While game level generation is not a particularly new field of
research, Super Mario World provides a more complex environment than
any related work has attempted to generate for. Also, this work uses
both more and more recent models, focusing especially on the
transformer architecture~\cite{vaswaniAttentionAllYou2017}, originally
introduced as a encoder-decoder model for neural translation tasks.
Due to its functional similarity to the long short term memory
architecture~\cite{hochreiterLongShorttermMemory1997} and its faster
training time, we believe it is a good fit for more challenging
generation tasks such as provided by Super Mario World.

\subsection{Games and Optimization Problems}

Humans have been playing games for a variety of reasons including
entertainment, socialization and education for over thousands of
years~\cite{HistoryGames2019}. They are an integral part to both human
culture, society and problem specific understanding. In modern,
recent, times, games and their theory have had an impact on other
fields of research as well. John Nash
proved~\cite{NashEquilibrium2019} that finite games with multiple
players have an optimal strategy, the Nash equilibrium. It is now an
essential part of economics and politics as
well~\cite{NashEquilibrium2019}; most problems with strategic
interaction between several agents can theoretically be solved so that
every agent gets the maximum reward based on every other agent's
optimal decision.

The video game industry in particular has a large impact on the
economy around the world. In~2018, the US video game industry even
matched the US film industry, a much older and more mainstream
industry, in revenue~\cite{VideoGameIndustry2019}. Possibly in part
due to its economic impact, the video game industry has become
responsible for many advances in technology, improving and developing
techniques and hardware~\cite{VideoGameIndustry2019} which have now
become the norm in many other fields as well due to their
effectiveness. A very practical and relevant example is the use of
graphical processing units (GPUs) to speed up the training of machine
learning models.

We have established that games are playing an important part in
society, furthering research and innovation in various areas and
giving people the ability to learn and socialize in a comfortable
environment. Due to the infinite possibilities in game environments,
an enjoyable game exists for almost everyone~-- similar to how almost
every person finds at least one music genre they enjoy. \\
However, while playing a game to optimize its underlying mathematical
problem is fun and important, in this work, we will focus on the
second part, research and innovation in the hopes of leveraging the
new knowledge in other fields. While we also optimize on a problem,
that problem is not related to a game itself but rather it is related
on a meta level. We try to generate platforming levels, optimizing
them to be similar to existing levels known to the system. Our system
will analyze a dataset of levels and try to generate new levels based
on the seen patterns. It will work without any prior knowledge or
feature engineering done by a human.

%%% Local Variables:
%%% mode: latex
%%% TeX-master: "../SMWLevelGenerator"
%%% End:

