\subsection{Generation via Prediction}
\label{sec:generation-via-prediction}

We now focus on implementation details and design decisions regarding
our sequence prediction models. In the following explanations, we will
assume the level is read column per column. The only difference when
reading per tile is the input size. While each of our models may have
different input preferences for a sequence (vector of vectors, matrix,
\dots), we will focus our explanation on individual sequence elements
as these all follow the same principles. Each sequence element is a
vector consisting of
\begin{itemize}
\item a constant input part containing metadata,
\item a bit that indicates whether the level has not ended yet (so it
  is~1 if the level has not ended yet and~0 if this is the last column
  or beyond), and
\item a column (or single tile) of the level.
\end{itemize}
The constant input part supplies our model with the following
metadata:
\begin{itemize}
\item level number
\item amount of screens
\item level mode
\item level graphics (remember that these influence interactivity as well)
\item main and midway entrance action (these determine whether the
  level is a water level)
\item sprite buoyancy (how sprites interact with water)
\item disable layer~2 buoyancy (see above but for interaction with layer~2 data)
\item sprite mode
\end{itemize}
As mentioned, this list is easily extensible if desired as our
foundation contains and parses all relevant information.

To support many different models, we implement an abstract model class
(also called the ``model interface'') that requires extending only a
few methods for new models (if the provided defaults are not already
correct). One of these methods solves the problem of the desired input
for the model. Our data iterator will automatically shape the input
correctly for each model. The currently supported input types are
vectors of vectors and matrices. Single, long vectors containing the
whole sequence are implemented but some questions about how their data
is ordered must be answered individually (for example, is the constant
input provided only once or for each column like for the other input
types).

We implement a second loss function, the ``soft'' loss, that assigns a
lower penalty (or none) to sequence elements that were predicted
incorrectly if the prior two sequence elements were the same. With
this, we hope to de-regularize our models so they are less heavily
influenced by sudden changes in the level. The user may choose to use
either one during training (see section~\ref{sec:training-loops}).

By default, our LSTMs use the rectified linear unit activation
function~\cite{nairRectifiedLinearUnits,RectifierNeuralNetworks2019}
for hidden layers. Leaning on the original
implementation~\cite{OpenaiGpt22019}, we use the Gaussian error linear
unit activation function~\cite{hendrycksGaussianErrorLinear2018} for
the transformers. \\
It is important to note that our sequence prediction models do not
train on their own generated data which may improve results (this
could be done with a multitude of different steps into the future).
This is a large simplification heavily influencing any prediction
results on predicted data. As all the errors for each prediction will
accumulate, results will suffer accordingly.

The transformer models suffer from an issue in \\
\mbox{Transformers.jl}~\cite{peterChengchingwenTransformersJl2019} due
to the way batching works for these models (fixing this was out of the
scope of this work): We obtain a complete level matrix, a singular
input, by concatenating the preprocessed columns in the second
dimension; we then concatenate multiple of these matrices in the third
dimension to get a batch of inputs. As neither Julia nor
\mbox{cuSPARSE} provides support for sparse arrays in the third
dimension, we are not able to obtain perfect performance for the
transformer models (as we mentioned, \mbox{cuSPARSE} could not be used
as of yet so this and the following only affects data on the CPU at
the moment). \\
This batching issue also has a large effect on memory usage as the
non-sparse data takes up a lot of space in high dimensions. For
3-dimensional data, a batch of 32~matrices of maximum length
(512~columns) would take
$58438 \cdot 32 \cdot 512 \cdot 2\text{ bytes} \approx 2 \cdot 10^{9} = 2\text{ GB}$ of data
(the first factor is the size of each columnar sequence element) (we
need 2~bytes per value due to the constant metadata inputs containing
16-bit values). Mapping this amount of data to a GPU that already
contains a large model is only possible in rare cases, so reducing
batch size, thereby decreasing efficiency, is the only option. In the
future, we will implement a wrapper around sparse matrices to act as a
3-dimensional array during computation for transformer models. \\
Even with these issues, our transformer models still mostly kept up
with or even outperformed the LSTM stacks, though this may be due to
inexperience on our side in choosing the correct amount of parameters.

We will now list the configuration options for our sequence prediction
models. As already mentioned, we provide convenience functions for
model creation for each training data dimensionality. As these are
only for convenience, they have some limitations\footnote{Obviously,
  any unlisted parameters can be changed manually by modifying the
  source code as well. This also goes for the models in the following
  sections.}.

LSTMs are built with the following parameters:
\begin{itemize}
\item amount of hidden layers
\item amount of hidden neurons per hidden layer
\item optional skip (or ``shortcut'')
  connections~\cite{heDeepResidualLearning2015} from each hidden layer
  directly to the final activation function
\item probability of dropout between every two hidden layers starting
  after the first one
\item final activation function
\end{itemize}
Our transformer creation functions take the following parameters:
\begin{itemize}
\item number of attention heads
\item hidden size of each attention head
\item hidden size of each fully connected layer
\item amount of transformer blocks
\item final activation function of each block
\item probability of dropout after attention layers and after each
  block
\item probability of dropout after fully connected layers
\item final activation function
\end{itemize}
The random model has a single parameter~-- its activation chance. By
default, it is set to the mean over the dataset for the appropriate
dimensionality (meaning the loss is optimal). \\
The input and output sizes are also adjustable for each model.

Having described the implementation details for our sequence
prediction models, we will now do the same for our GANs.

%%% Local Variables:
%%% mode: latex
%%% TeX-master: "../../SMWLevelGenerator"
%%% End:

