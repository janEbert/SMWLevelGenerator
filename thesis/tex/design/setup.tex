\subsection{Setup}

With our code living in a Git~\cite{Git} repository, we want to
achieve~-- aside from the obvious version control advantages~-- an
easily reproducible but also portable environment (by which we mean no
large binary files). We provide several individual scripts for the
whole setup pipeline of our environment, including the following:
\begin{itemize}
\item downloading the required tools (optionally also additional ones)
\item downloading the properly filtered hacks (as we established in
  section~\ref{sec:hacks}, these are in patch form)
\item decompressing them to properly named directories (supporting
  either \texttt{tar} or \texttt{7z} as desired)
\item applying the patches to the original ROM
\item dumping all levels in each ROM
\end{itemize}
All of these are combined in a single convenience script the user can
execute in order to obtain their own environment. We specify a default
directory structure our source code is based on so no paths have to be
changed if using the convenience script. Note that some patches, ROMs
and levels cause issues; we mentioned that these are listed in the
\texttt{stats} directory in section~\ref{sec:hacks}.

Some time after our download of the over 300~hacks, SMW Central
implemented protection against too many requests in a certain time
frame. Even though we have set a generous wait time limit between
downloads, we understand that troubling the servers over and over for
the same files may not be desired. Due to this, we supply scripts that
fetch the files (tools and hacks) from their original source, SMW
Central, and scripts that download the files collected in a custom
location provided by us (the convenience setup script exists in both
versions).

While we mentioned portability, the scripts above are currently only
implemented in shell script\footnote{Bash compatibility, aiming for
  POSIX.}. For cross-platform support, these will be implemented in
Julia as well at a later time, eliminating another dependency. \\
For people either not interested in a manual setup or without access
to a shell script interpreter, we provide downloads for pre-computed
databases in the dimensions we support out of the box (listed in
section~\ref{sec:preprocessing}). If, on the other hand, the levels
are available, generating those databases is done in a single function
call (\texttt{generate\_default\_databases()}).

Our repository contains pre-calculated values for several statistics
(these exist mostly to minimize the amount of layers we supply to the
models). However, when setting up manually, we recommend running the
functions \texttt{SMWLevelGenerator.Sprites.generateall()} and
\texttt{SMWLevelGenerator.SecondaryLevelStats.generateall()} to
pre-calculate statistics based on the user's dataset. If any of the
files containing statistics is not found, that statistic is otherwise
calculated on each load of the module. Depending on the size of the
dataset, this make take some time, therefore, pre-calculating will
speed up precompilation of the module.
\medskip

Several other scripts allow further processing of the data, including:
\begin{itemize}
\item summarizing statistics of the hacks (like authors, rating,
  \dots)
\item removing duplicate and test levels based on their CRC32
  checksums
\item removing levels with non-vanilla behavior
\end{itemize}
These have already been implemented in Julia.

Detailed setup instructions are found in the project repository's
README.

%%% Local Variables:
%%% mode: latex
%%% TeX-master: "../../SMWLevelGenerator"
%%% End:

