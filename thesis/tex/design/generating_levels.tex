\subsection{Generating Levels}
\label{sec:generating-levels}

With the GAN outputs and the following metadata, the sequence
predictors have all the information they would get from a ``real''
level. Applying the models in sequence is one step; then we need to
reverse the whole preprocessing pipeline, converting our abstract
representation back to the correct formats that Lunar Magic can write
into the ROM. In this section, we will step through this pipeline one
by one.

% TODO reference to figure of pipeline (should be somewhere previous)
We start with random input of the correct size for our generator,
apply it and get our first screen. At this point, we could also apply
our discriminator and retry generating the first screen if the emitted
value is too low. We have not implemented this feature. \\
The first screen is then fed into the metadata predictor, giving us
the generated level's metadata. This constant input is then combined
with the level in the correct specification, depending on whether the
sequence predictor accepts columns or singular tiles, and all but the
last sequence elements have their ``sequence has not ended yet''-bit
set to 1. \\
The data is now in the same format as a preprocessed real level and we
apply the sequence prediction model sequentially until we either find
that the ``sequence has not ended yet''-bit is 0 or until the maximum
level length is reached. This complete output is then postprocessed.

As we skipped some metadata when preparing our data, we need to fill
the omissions with default values. For all unknown data, we chose
either the defaults of level 261, the first level Lunar Magic presents
to the user, or give ``empty'' values if possible. Before we give an
example, remember that Lunar Magic dumps sprite data in a ``.sp'' file
containing the sprite header and then data of individual sprites in
the level until a designator for the ending is encountered. If our
model does not generate sprites, we don't write back the sprites in
level 261 but instead set the sprite header to that of level 261 but
do not give any sprites themselves. While the sprite header is
required, sprites are not.

Generation functions include generating a single level or multiple
levels. A function to generate a whole hack of levels (meaning 512
different levels, each with a different number) is planned. We also
implement model-specific generation functions. For the sequence
prediction model, functions may generate levels like the above
functions or based on predicting on the original game's levels. The
GANs may also generate only the first screen for a level
independently.

Note that during all of this, the user does not have to worry about
dimensionality in any way; all of this is handled automatically if the
model was supplied the correct parameters (such as its
dimensionality). We allow writing back 1-dimensional outputs by
hardcoding the default row the generated output should be placed in
and by filling the rest of the level with empty tiles.

We will now dive into implementation details and design decisions
regarding our training pipeline, the most important part of our
program that first enables our models to generate meaningful outputs.

%%% Local Variables:
%%% mode: latex
%%% TeX-master: "../../SMWLevelGenerator"
%%% End:

